\thispagestyle{fancy}
\begin{center}
	\LARGE{\textbf{Circuitos Serie-Paralelo}}
\end{center}
\section{Objetivos}
Al finalizar esta experiencia, usted estará capacitado para:
\begin{enumerate}
	\item Calcular la resistencia equivalente de circuitos mixto (serie-paralelo).
	\item Usar el DMM para medir la resistencia equivalente  de circuitos mixto (serie-paralelo). 
\end{enumerate}
\section{Conocimientos previos}
La mayoría de los circuitos resistivos pueden ser convertidos a circuitos mas sencillos reemplazando las combinaciones de resistores en serie o en paralelo por sus resistencias equivalentes.
Para ello, se reemplaza metódicamente toda combinación serie o paralelo  por una única resistencia equivalente. Este procedimiento se repite hasta que todas las combinaciones paralelo y serie  hayan sido eliminadas. De esta manera, los circuitos  serie-paralelo pueden ser reducidos a circuitos simples. 
\section{Equipo}
El siguiente equipo es necesario para la realización del experimento.
\begin{enumerate}
	\item Módulo de experimentos.
	\item DMM (Multímetro digital).
\end{enumerate}
\section{Procedimiento}
\begin{enumerate}
	\item Registre los códigos de colores de los resistores $R_{1}, R_{2}, R_{3}, R_{4}, R_{5}, R_{6}, R_{7}, R_{8}$ en el cuadro 1.
	\begin{table}[h]
		\centering
		\begin{tabular}{|c|c|}
			\hline
			Resistor & Valor\\ \hline
			$R_{1}$&0.1k\\ \hline
			$R_{2}$&0.47k\\ \hline
			$R_{3}$&0.68k\\ \hline
			$R_{4}$&0.27k\\ \hline
			$R_{5}$&22k\\ \hline
			$R_{6}$&27k\\ \hline
			$R_{7}$&47k\\ \hline
			$R_{8}$&39k\\ \hline
		\end{tabular}
		\caption{}
	\end{table}
	\item Ahora, convierta la red conformada por los resistores $R_{1}$ al $R_{4}$  a sus resistores equivalente utilizando el siguiente método:
	\begin{enumerate}
		\item Convierta en$R_{2}$ y $R_{3}$ (en paralelo) en una sola resistencia equivalente.
		\\ $R_{2}$  en paralelo con $R_{3}$ es igual 277.91 
		\item El circuito ahora contiene en serie a $R_{1}$, la resistencia calculada anteriormente  y  $R_{4}$.
		\\ Resistencia equivalente es igual a 647.91
	\end{enumerate}
	\item Convierta la red conformada por los resistores $R_{5} $ al $R_{8}$ a sus resistores equivalentes utilizando el siguiente método:
	\begin{enumerate}
		\item Calcule el resistor equivalente de $R_{6}$ y $R_{8}$.
		\\ El valor equivalente de $R_{6}$ y $R_{8}$ es igual 66$k\Omega$
		\item Esta nueva resistencia está en paralelo con $R_{7}$. Calcule el valor de esta combinación.
		\\ $R_{7}$ en paralelo con el valor anterior es igual a $27.45k\Omega$
		\item Este valor está en serie con $R_{5}$. Calcule la resistencia total equivalente de toda la red.
		\\ La resistencia equivalente de la red es igual a $49.45k\Omega$
	\end{enumerate}
	\item Mida con el DMM la resistencia equivalente de la red formada por $R_{1}$ hasta $R_{4}$.
	\\ Resistencia equivalente $R_{1} a R_{4}$ es igual a $467.91k \Omega$
	\item Mida resistencia equivalente de la red que consta de $R_{5} a R_{8}$ .
	\\ Resistencia equivalente $R_{5} a R_{8}$  es igual a $49.5411 K \Omega$
	\item Compare sus mediciones con sus cálculos. Si la diferencia entre ambos es mayor del $10\%$, consulte a su profesor.
	\begin{table}[htbp]
		\centering
		
		\begin{tabular}{|c|c|c|}
			\hline
			\multirow{2}{*}{Red de resistores} & \multicolumn{2}{c|}{Valor} \\
			\cline{2-3}
			& Calculado & Medido \\
			\hline
			$R_{1} - R_{4}$ &647.9k  & 647.69k \\
			\hline
			$R_{5} - R_{8}$ &49.45k &49.9k \\
			\hline
		\end{tabular}
	\end{table}
	\\ Esta es la parte del experimento, el circuito sufrirá una modificación.
	\item El circuito ($R_{1}, R_{2}, R_{3} $ y $R_{4}$) ha sufrido modificado.
	\\ Ubique los resistores donde el cambio tuvo lugar midiendo resistencia con le DMM.
	\\ Nueva resistencia total es igual a 
	\\$R_{1}, R_{2}, y R_{4}$ están conectadas en serie, y $R_{3}$ fue desconectado del circuito. ¿Qué cambio se introdujo en el circuito?
	\item EL circuito formado por $R_{5}, R_{6}, R_{7}$ y$ R_{8}$ ha sido modificado.
	\\ Ubique el lugar donde el cambio tuvo lugar midiendo resistencias con el DMM. ¿Qué cambio se introdujo en el circuito?
	\\Cuando quitamos la resistencia R13 en paralelo, ahora la resistencia equivalente será la suma de las resistencias R11, R12 y R14.
	\item EL circuito formado por  $R_{5}, R_{6}, R_{7}$ y$ R_{8}$ ha sido modificado. 
	\\ Cuando se quitó la resistencia R17 del circuito, pudimos visualizar que la resistencia equivalente aumento a lo del anterior.
	\\ Ubique el lugar donde el cambio tuvo lugar midiendo resistencias con el DMM. \\ La nueva resistencia total es igual a $88k\Omega$
\end{enumerate}
\section{Conclusiones}
En conclusión, podemos decir que es necesario conocer los códigos de colores para así saber el valor de la resistencia que se está trabajando y también saber si las resistencias están conectadas en paralelas o serie ya que la resistencia equivalente varia según se ha el caso.